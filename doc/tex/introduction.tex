\chapter{Introduction}

This paper describes a system, designed for controlling the moisture level of a plant. It reads the data from a moisture sensor and based on it and on the configuration parameters, it enables a water pump which irrigates the plant. The system contains an integrated web server which allows the user to configure the parameters of the controller. By using RESTful endpoints it provides the possibility of communication with a wide range of clients such as mobile devices, other web servers or front-ends.

\paragraph{Motivation} The reason for choosing this project starts from a personal experience in the past. My parents left on my care all the plants in our house to irrigate them regularly. I did not forgot the task, and executed it every week. However after they came back our coffee tree lost all of its leaves because of dehydration. Then I understood that it requires more frequent irrigation. Yet the other plants were in a good condition because the weekly irrigation was just right for them. This made me realize that each plant has its own necessities and the owner should take them into consideration. But this requires effort and time which are priceless when one must focus all of them to learn for the exams. Therefore I thought about automatizing the irrigation process and make it configurable in order to satisfy the needs of each individual plant.

\label{sec:not_new}
The concept of an automatized irrigation station is not novel. There are existing implementations for both industrial and commercial uses. However the domain of application is really large. There are significant differences between the implementations, based on the use cases, precision and available technology. This paper describes the solution which can satisfy the needs of a specific set of use cases.
